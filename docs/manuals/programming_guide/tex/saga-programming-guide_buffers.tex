
\subsection{Quick Introduction}

 Various classes (e.g.,  \T{saga::file} and \T{saga::stream}) in the SAGA
 API expose I/O operations, i.e., chunks of binary data can be written
 to, or read from these classes.  Other classes (such as \T{saga::rpc})
 handle binary data as parameters.  In order to unify the application
 management of these data, SAGA introduces the \T{saga::buffer} class,
 which is essentially a simple container class for a byte buffer, plus
 a number of management methods.  Various subclasses of the
 \T{saga::buffer} exist, and, as described below, users are allowed, and
 actually encouraged, to build their own ones.

 The C++ rendering of SAGA distinguishes between mutable and
 non-mutable buffers: non-mutable buffers are used for write-type 
 operations, and cannot be changed by the SAGA implementation; mutable
 buffers are for read-type operations, and can be changed (i.e., new
 data can be added to the buffer).


\subsection{Reference}

 \begin{mycode}[label=Prototype: saga::buffer]
  namespace saga 
  {
    class const_buffer
        : public saga::object
    {
      public:
        const_buffer  (void const  * data, 
                       saga::ssize_t size);
        ~const_buffer (void);

        saga::ssize_t  get_size (void) const;
        void const *   get_data (void) const;
        void           close    (double timeout = 0.0);
    };

    class mutable_buffer
        : public saga::const_buffer
    {
      public:
        typedef void buffer_deleter_ type(void* data);

        typedef TR1::function <buffer_deleter_type> buffer_deleter;
        static void default_buffer_deleter (void * data);

        mutable_buffer  (saga::ssize_t  size = -1);
        mutable_buffer  (void * data, 
                         saga::ssize_t  size);
        mutable_buffer  (void * data, 
                         saga::ssize_t  size, 
                         buffer_deleter cb);
        ~mutable_buffer (void);

        void   set_size   (saga::ssize_t  size = -1);
        void   set_data   (void * data, 
                           saga::ssize_t  size,
                           buffer_deleter cb = default_buffer_deleter);
        void * get_data   (void);  // non-const version
    };
  }
\end{mycode}


 \subsection{Details}
 
  Although the concept of an I/O buffer is very simple, and the
  prototype shown above is rather straight forward, the semantic
  details of the SAGA buffer are relatively rich.  That holds true in
  particular for the memory management of the buffer data segment.
  For the interested reader, the |saga::buffer| section in the SAGA
  Core API specification contains quite some detail on that issue.

  \subsubsection{Buffer Memory Management}

   In general, buffers can operate in two different memory
   management modes: the data segment can be user-managed (i.e.,
   application-managed), or SAGA-managed (i.e., implementation-managed).  The constructors allow the application to pass a memory
   area on buffer creation: if that buffer is given, and not-NULL,
   then the SAGA implementation will use that buffer, and will never
   re- nor de-allocate that memory (memory management is left up to
   the application).  On the other hand, if that memory area is not
   given, or given as NULL, then the SAGA implementation will
   internally allocate the required amount of memory, which MUST NOT
   be re- or de-allocated by the application (memory management is
   left to the SAGA implementation).

   Although the latter version is certainly convenient for the end
   user, it comes with a potential performance penalty:  data from the 
   implementation allocated buffer may sometimes need an additional
   memcopy into application memory.  If that is the case, it is up to you to decide what  memory management mode works best for your application use
   case.

   \HINT{The most performant case is most of the time to re-use a
   single (or a small set of) application allocated memory buffer(s)
   over and over again.  Note that you can use a larger size memory
   segment for a small buffer by giving a smaller size parameter to
   the constructor.}

 \begin{mycode}[label=Example saga::buffer usage]
  // allocate a 'large' buffer statically
  char mem[1024];

  // create a saga buffer object for reading (mutable)
  saga::mutable_buffer buf (mem, 512);

  // open a file
  saga::url u ("/etc/passwd");
  saga::filesystem::file f (u);

  // read data into buffer - the first 512 bytes get fill
  f.read (buf);

  // seek the buffer, so that the next read goes into the
  // second half of the buffer
  buf.set_data (mem + 512, 512);

  // now read again
  f.read (buf);

  // the complete buffer should be filled now
  // print what we got
  std::cout << mem << std::endl;
 \end{mycode}


   \subsubsection{Const versus Mutable Buffers}

   On |write|-like operations, the SAGA implementation has no need to
   change the buffer's data segment in any way: it only needs to read
   the data, and to copy them to whatever entity the write operations
   happens upon.  The implementation can thus treat the buffer as
   |const|, which allows a number of optimizations and memory access
   safeguards to be employed.  

   On the other hand, |read|-like operations will usually require the
   SAGA implementation to write, or even to (re-)allocate the buffers
   memory segment.  In such cases, |const| safeguards cannot be
   employed.  

   \HINT{It is encouraged the use of \T{const\_buffer} instances for
   \T{write}-like operations, and of \T{mutable\_buffer} instances for
   \T{read}-like operations.}

   In order to simplify memory management and to provide optimal
   memory access safeguards, the SAGA C++ bindings distinguish between
   |const_buffer| and |mutable_buffer| classes.  Both types can be
   used for |write|-like operations, but only |mutable_buffer|
   instances can be used for |read|-like operations.
   
   
   \HINT{It is possible to case \T{mutable\_buffer} instances to
   \T{const\_buffer}, which allows to re-use buffers for all I/O
   operations and, at the same time, allows the implementation to use
   \T{const} checking.\\ 


