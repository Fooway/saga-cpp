
 \subsection{Quick Introduction}

  Namespaces are used for a wide variety of computer
  subsystems: they are used to organize files, to manage domain
  names, web content, etc.  Hierarchical namespaces are
  prevalent for managing files (indeed, most file systems provide
  a hierarchical namespace, the directory tree).  SAGA provides
  the |saga::name_space| package to navigate and manipulate such
  namespaces:

  \begin{mycode}[label=Navigating a name space]
  #include <saga/saga.hpp>

  int main (int argc, char** argv)
  {
    // open a namespace directory
    saga::url u (std::string ("file://localhost/") + getenv ("PWD"));
    saga::name_space::directory d (u);

    // list the contents
    std::vector <saga::url> entries = d.list ();

    // print the entries, and their type
    for ( unsigned int i = 0; i < entries.size (); i++ )
    {
      std::string type;
      // get some details for the entry
      if ( d.is_dir (entries[i]) )
      {
        type = "/";
      }
      // if a link (symbolic)
      else if ( d.is_link (entries[i]) )
      {
        type = " -> " + d.read_link (entries[i]).get_string();
      }
      // print the info
      std::cout << entries[i] << type << std::endl;
    }

  return (0);
  }
  \end{mycode}
   
  This is a poor man's ls!  It needs only five SAGA calls, which
  are all very simple.  Note that a directory has a `cwd', a
  Current Working Directory.  Calling |cd()| on a directory
  changes that.  Relative file names are always interpreted with
  respect to that cwd.

  So, the |saga::name_space| package provides two classes:
  |directory| and |entry| (note that directory inherits entry).
  On entries, you can perform |copy()|, |link()|, |move()|, and
  |remove()|.  You can also inspect entries, with |is_dir()|,
  |is_entry()|, |is_link()| and |read_link()|.  

  Directories provide the same operations, but with an additional
  |source| argument, as shown below:

  \shift |entry.copy (target);          // copy entry      to target|\\
  \shift |  dir.copy (target);          // copy dir        to target|\\
  \shift |  dir.copy (source, target);  // copy dir/source to target|

  But the directory copy should be recursive, of course:

  \shift |  dir.copy (target, saga::name_space::Recursive);|

  Additionally, the directory has two open methods:
  
  \shift |saga::entry     e = dir.open     (saga::url (entry_name));|\\
  \shift |saga::directory d = dir.open_dir (saga::url (dir_name)  );|


 \subsection{Reference}
 
  \begin{mycode}[label=Prototype: saga::namespace::flags]
   namespace saga
   {
     namespace name_space 
     {
       enum flags
       {
         Unknown         =  -1,
         None            =   0,
         Overwrite       =   1, 
         Recursive       =   2,
         Dereference     =   4,
         Create          =   8,
         Exclusive       =  16,
         Lock            =  32,
         CreateParents   =  64
       };
    }
  }
  \end{mycode}
 
 
  \begin{mycode}[label=Prototype: saga::namespace::entry]
   namespace saga
   {
     namespace name_space 
     {
       class entry
           : public saga::object,
             public saga::monitorable,
             public saga::permissions
       {
         entry  (session const & s, 
                 saga::url       url, 
                 int             mode = None);
         entry  (saga::url       url, 
                 int             mode = None);
         entry  (void);
         ~entry (void);
   
         // inspection methods
         saga::url get_url   (void) const;
         saga::url get_cwd   (void) const;
         saga::url get_name  (void) const;
         saga::url read_link (void) const;
 
         bool      is_dir    (void) const;
         bool      is_entry  (void) const;
         bool      is_link   (void) const;
 
         // management methods
         void      copy      (saga::url target, 
                              int flags = saga::name_space::None);
         void      link      (saga::url target, 
                              int flags = saga::name_space::None);
         void      move      (saga::url target, 
                              int flags = saga::name_space::None);
         void      remove    (int flags = saga::name_space::None);
         void      close     (double timeout = 0.0);
       };
     }
   } 
 
 \end{mycode}
 
  \begin{mycode}[label=Prototype: saga::namespace::directory]
   namespace saga
   {
     namespace name_space 
     {
       class directory 
           : public saga::name_space::entry
       {
         public:
           directory  (session const & s, 
                       saga::url       url, 
                       int             mode = None);
           directory  (saga::url       url, 
                       int             mode = None);
           directory  (void);
           ~directory (void);
 
           void        change_dir (saga::url    target)
           std::vector <saga::url> 
                       list       (std::string  pattern = "*", 
                                   int          flags   = None) const;
           std::vector <saga::url> 
                       find       (std::string  pattern, 
                                   int          flags   = Recursive) const;
 
           saga::url   read_link  (saga::url    url) const;
           bool        exists     (saga::url    url) const;
           bool        is_dir     (saga::url    url) const;
           bool        is_entry   (saga::url    url) const;
           bool        is_link    (saga::url    url) const;
           unsigned    int get_num_entries 
                                  (void) const;
           saga::url   get_entry  (unsigned int entry) const;
           void        copy       (saga::url    source_url, 
                                   saga::url    dest_url, 
                                   int          flags = None);
           void        link       (saga::url    source_url, 
                                   saga::url    dest_url, 
                                   int          flags = None);
           void        move       (saga::url    source_url, 
                                   saga::url    dest_url, 
                                   int          flags = None);
           void        remove     (saga::url    url, 
                                   int          flags = None);
           void        make_dir   (saga::url    url, 
                                   int          flags = None);
           entry       open       (saga::url    url, 
                                   int          flags = None);
           directory   open_dir   (saga::url    url, 
                                   int          flags = None);
       };
     } // namespace_dir
   } // namespace saga
 
 \end{mycode}


 \subsection{Details}

  The notion of a namespace is shared by several SAGA packages: the
  file, the replica and the advert packages. All allow to manage
  entities which are organized in a hierarchical namespace.  The
  namespace package is abstracting the management of that hierarchy,
  and leaves only the entity specific operations (File-IO, Replica
  Management, Advert Creation) to the various packages.

  \HINT{Namespace operations handle namespace entries as opaque, and
  are never able to look 'inside'.  To do that, use one of the
  derived packages, which add exactly that functionality, specific to
  the type of namespace they represent.}
  
  The namespace package introduces two classes:
  |saga::namespace::entry| and |saga::namespace::directory| (which is
  an entry, i.e., inherits from |entry|).  Both classes refer to
  entities which are specified by a pathname, typically a URL.
  Several calls additionally allow to refer to entries with wildcards,
  similar to the shell wildcards known by POSIX (see |glob(7)|)
  \footnote{Namespace entries can also have permissions, just as
  in most file systems.  For details on permissions, see
  Section~\ref{sec:permissions}.}.

  The namespace package allows to create and delete entries, to copy,
  move or link entries, and to inspect entries and directories.

  % \X{add infos about available attributes and metrics}

