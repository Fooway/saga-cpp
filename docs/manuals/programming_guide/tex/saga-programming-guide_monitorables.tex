
 \subsection{Quick Introduction}

  \subsubsection*{Monitorables and Metrics}

  \X{fill in}


  \subsubsection*{Notifications}

  Closely related to monitorables and metrics are notifications; they
  notify the application of certain events. For example, when a task
  is done, when it changes its state, etc.  For those events, the
  application can create and register custom callbacks:


  \begin{mycode}[label=Notifications on task state changes]
  #include <saga/saga.hpp>

  class my_callback : public saga::callback
  {
    public:
      // cb exits the program with appropriate error code
      bool cb (saga::monitorable mt,
               saga::metric      m,
               saga::context     c)
      {
        if ( m.get_attribute ("state") == "Done" ) {
          exit (0);
        }

        if ( m.get_attribute ("state") == "Failed" ) {
          exit (-1);
        }

        // all other state changes are ignored
        return true; // keep callback registered
      }
  }

  int main (int argc, char** argv)
  {
    // run a file copy asynchronously
    saga::url u (argv[1]);
    saga::filesystem::file f (u);
    saga::task t = f.copy <saga::task::Async> (saga::url (argv[2]));

    // monitor the task state
    my_callback cb;
    t.add_callback ("state", cb);

    // make sure the task finishes
    t.wait ();
  }
  \end{mycode}

  The example above registers a private callback to the |"state"|
  metric of the task.  In fact, many SAGA objects (which implement the
  monitoring interface, and are thus |monitorable|s) have several
  metrics, which can be queried by |monitorable.list_metrics()|.

  \HINT{The return value of the callback determines if the callback
  stays registered (\T{true}) or not (\T{false}).  This allows for a
  \I{call-once} semantics.}


 \subsection{Reference}


 \subsection{Details}


