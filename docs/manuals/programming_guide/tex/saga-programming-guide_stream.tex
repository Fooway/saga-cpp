

 \subsection{Quick Introduction}

  \B{(FIXME: english)}\\
  Inter-process communication (IPC) is historically central to
  parallel and distributed computing, and numerous approaches exist
  for the wide variety of use cases.  Streams are amongst the most
  well-known and most simple of these paradigms, and are also provided
  by SAGA.  The concept is very close to that of BSD sockets: a
  |stream_service| can be listened to (calling |serve()|), and a
  connection client |stream| results in a |stream| instances on both
  ends, upon which the application can call |read()| and |write()|:


  \begin{mycode}[label=Stream server]
  #include <saga/saga.hpp>

  int main (int argc, char** argv)
  {
    // reusable io buffer
    saga::buffer buf;

    // create a stream_server and listen for clients
    saga::stream::server ss (saga::url (argv[1]));

    while ( 1 )
    {
      // wait for incoming client
      saga::stream::client s = ss.serve ();
      
      // read data from client
      s.read (buf);
    }
  }
  \end{mycode}
  \begin{mycode}[label=Stream client]
  #include <saga/saga.hpp>

  int main (int argc, char** argv)
  {
    // create client stream and connect to server
    saga::stream::client s (saga::url (argv[1]));

    s.connect ();

    // send some data
    s.write (saga::buffer ("Hello"));
  }
  \end{mycode}


  Note that SAGA is silent about the used wire protocol.  In
  particular, server and client are responsible for picking a compatible
  transfer mechanism.  Also, the boostrapping mechanism is out of
  scope in the SAGA stream package, i.e., both sides need to agree on
  the URL for the server endpoint out-of-band.  Both issues will be
  addressed in future SAGA API extensions (message API and advert API
  extension).

  In particular, for stream-based communication, asynchronous operations
  and notifications are extremely useful programming paradigms.  Both
  are provided within SAGA, more details can be found in the later
  description of the SAGA \LF (|saga::monitoring| and |saga::task|
  package).


 \subsection{Reference}

  \begin{mycode}[label=Prototypes: saga::stream]
  namespace saga
  {
    namespace stream 
    {
      namespace attributes 
      {
        char const * const stream_bufsize       = "Bufsize";
        char const * const stream_timeout       = "Timeout";
        char const * const stream_blocking      = "Blocking";
        char const * const stream_compression   = "Compression";
        char const * const stream_nodelay       = "Nodelay";
        char const * const stream_reliable      = "Reliable";
      }

      namespace metrics 
      {
        char const * const stream_state         = "stream.State";
        char const * const stream_read          = "stream.Read";
        char const * const stream_write         = "stream.Write";
        char const * const stream_exception     = "stream.Exception";
        char const * const stream_dropped       = "stream.Dropped";

        char const * const server_clientconnect = "server.ClientConnect";
      }

      enum state
      {
        Unknown      = -1,
        New          =  1,
        Open         =  2,
        Closed       =  3,
        Dropped      =  4,
        Error        =  5
      };

      enum activity
      {
        Read         =  1,
        Write        =  2,
        Exception    =  4
      };

      class stream 
          : public saga::object,
            public saga::attributes,
            public saga::monitorable
      {
        public:
          stream (saga::session const & s, 
                  saga::url             url = saga::url ());
          stream (saga::url             url);
          stream (void);
          stream (saga::object const  & o);
         ~stream (void);

          stream & operator= (saga::object const & o);

          saga::url     get_url     (void) const;
          saga::context get_context (void) const;

          saga::context connect     (void);

          std::vector <activity>
                        wait        (activity       what, 
                                     double         timeout = -1.0);
          void          close       (double         timeout =  0.0);

          saga::ssize_t read        (saga::mutable_buffer buffer, 
                                     saga::ssize_t        length  = 0);
          saga::ssize_t write       (saga::const_buffer   buffer, 
                                     saga::ssize_t        length  = 0);
      };

      class server 
          : public saga::object,
            public saga::monitorable,
            public saga::permissions
      {
        public:
        server (saga::session const & s, 
                saga::url            url = saga::url ());
        server (saga::url            url);
        server (void);
        server (saga::object const & o);
       ~server (void);

        server & operator= (saga::object const & o);

        saga::url            get_url (void) const;
        saga::stream::stream serve   (double timeout = 0.0);
        void                 close   (double timeout = 0.0);
      };
    }
  }
\end{mycode}


 \subsection{Stream Details}

  The SAGA streams package allows establishing the simplest
  possible authenticated socket connection, with hooks to
  support application level authorization, and encryption
  schemes.  The stream API has the following characteristics:

  \begin{enumerate}

    \item It is not performance-oriented:  If performance is
    required, then it is better to program directly against the
    API's existing performance-oriented protocols like GridFTP
    or XIO.  

    \item It does not attempt to create a programming paradigm
    that diverges very far from baseline BSD sockets, Winsock,
    or Java Sockets.

  \end{enumerate}

  This API greatly reduces the complexity of establishing
  authenticated socket connections in order to communicate with
  remotely located components.  However, it provides very
  limited functionality and is thus suitable for applications
  that do not have very sophisticated requirements (as per 80-20
  rule).  It is envisaged that as applications become
  progressively more sophisticated, they will gradually move to
  more sophisticated native APIs in order to support those
  needs.  Later SAGA versions may offer higher level
  communication abstractions, such as messages.

  \subsubsection{Endpoint URLs}

   The SAGA stream API uses URLs to specify connection
   endpoints.  These URLs are supposed to allow SAGA
   implementations to be interoperable.  For example, the URL

      \shift |tcp://remote.host.net:1234/|

   is supposed to signal that a standard |tcp| connection can be
   established with host |remote.host.net| on port |1234|.  No
   matter what the specified URL scheme is, the SAGA stream API
   implementation MUST have the same semantics on an API level, i.e.,
   behave like a reliable byte-oriented data stream.

  \subsubsection{Usage}

   Just as for BSD sockets, a stream communication channel is
   established by creating a serving part (BSD: listening
   socket), and a client party (BSD: connecting socket).  After
   connecting both parties, reading and writing on the stream
   allows to exchange data.

  \begin{mycode}[label=Stream example - server side]
  { 
    // set up the serving endpoint on port 1234
    saga::stream::server server (saga::url ("tcp://localhost:1234"));

    // wait for incoming connections
    saga::stream stream = server.serve ();

    // if one arrived, greet the client
    stream.write (saga::buffer ("Hello World!", 13));
  }
  \end{mycode}

  \begin{mycode}[label=Stream example - client side]
  { 
    // setup stream for the server endpoint
    saga::stream::stream (saga::url ("tcp://remotehost:1234"));

    // connect to the server
    stream.connect ();

    // read the greeting message
    saga::buffer buf (13);
    stream.read (buf);

    // print it
    std::cout << buf.get_data () << std::endl;
  }
  \end{mycode}

  % \X{add details about async ops, and AAA}

