
 \subsection{Quick Introduction}

  All functional SAGA API calls come in three flavours:
  synchronous, asynchronous, and task versions.  The easiest way to
  understand the relations and differences between them is to
  consider that \I{all} functional API calls are represented by
  stateful |tasks| (e.g., a |file.copy()| API call is a task which
  can be in |Running| or |Done| state).

  \myfig{task_states}{\footnotesize The SAGA task state model.}

  Figure~\ref{fig:task_states} shows the available task states, and the various possible state transitions.  The different
  SAGA API call invocations are simply representing ways to create the
  respective tasks in different states: Synchronous calls create tasks
  in a final state (|Done| or |Failed|); Asynchronous operations
  create tasks which are |Running|; and the Task version of the calls
  creates tasks which are |New|, and need to be |run()| to do anything
  at all:


  \begin{mycode}[label=Asynchronous operations]
  int main (int argc, char** argv)
  {
    // create a file instance
    saga::url u (argv[1]);
    saga::filesystem::file f (u);

    // target dir to copy file to
    saga::url tgt ("file://localhost/tmp");

    // run file copy in three flavours
    saga::task t_1 = f.copy <saga::task::Sync>  (tgt);
    saga::task t_2 = f.copy <saga::task::Async> (tgt);
    saga::task t_3 = f.copy <saga::task::Task>  (tgt);

    // the three tasks do the same, but t_1 is already done at
    // this point, t_2 could still be running, and t_3 did not yet
    // start.

    // let's get the async version done:
    t_2.wait ();

    // now, run and finish the Task version:
    t_3.run ();
    t_3.wait ();

    // all three copies are done here.
  }
  \end{mycode}

  For convenience, the synchronous version is provided simply as:

   \shift |f.copy (tgt);|

  Now, what about the return values?  For those, the tasks provide a typed
  member method, |get_result()|, which can only be called when a task
  is in a final state:
  
  \begin{mycode}[label=Return values for asynchronous operations]
  int main (int argc, char** argv)
  {
    // create a file instance
    saga::url u (argv[1]);
    saga::filesystem::file f (u);

    // run method in three flavours
    saga::task t_1 = f.get_size <saga::task::Sync>  ();
    saga::task t_2 = f.get_size <saga::task::Async> ();
    saga::task t_3 = f.get_size <saga::task::Task>  ();

    // let's get the async version done:
    t_2.wait ();

    // now, run and finish the Task version:
    t_3.run ();
    t_3.wait ();

    // all three calls are done here.

    ssize_t size;
    size = t_1.get_result <ssize_t> ();
    size = t_2.get_result <ssize_t> ();
    size = t_3.get_result <ssize_t> ();

    
  }
  \end{mycode}
  
  The convenience of the synchronous version
  above is that it returns the return value immediately:

   \shift |ssize_t size = f.get_size ();|
  

 \subsection{Reference}



 \subsection{Details}


