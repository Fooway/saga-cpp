
\subsection{Quick Introduction}

 Attributes in SAGA are handled via the \T{saga::attribute} interface
 shown above.  That interface allows to set, query, and to inspect
 specific attributes on those SAGA objects that implement the
 interface.  
 

\subsection{Reference}

 \begin{mycode}[label=Prototype: saga::attribute]
  namespace saga
  {
    namespace attributes
    {
      // common attribute values
      char const * const common_true  = "True";
      char const * const common_false = "False";
    }
    
    class attribute 
    {
      public:
        typedef std::vector <std::string>              strvec_type;
        typedef std::map    <std::string, std::string> strmap_type;
  
        std::string  get_attribute          (std::string key)  const;
        void         set_attribute          (std::string key, 
                                             std::string val);

        strvec_type  get_vector_attribute   (std::string key)  const;
        void         set_vector_attribute   (std::string key, 
                                             strvec_type val);

        void         remove_attribute       (std::string key);

        strvec_type  list_attributes        (void) const;
        strvec_type  find_attributes        (std::string pat)  const;

        bool         attribute_exists       (std::string key)  const;
        bool         attribute_is_readonly  (std::string key)  const;
        bool         attribute_is_writable  (std::string key)  const;
        bool         attribute_is_vector    (std::string key)  const;
        bool         attribute_is_removable (std::string key)  const;
    };
  }
 \end{mycode}


\subsection{Details}

 A prominent example is the \T{saga::job::description}
 class:

 \begin{mycode}
  saga::job::description jobdef;

  std::vector <std::string> args;
  args.push_back ("2");
  
  jobdef.set_attribute        ("Executable", "/bin/sleep");
  jobdef.set_vector_attribute ("Arguments",  args);

 
  saga::job job = js.create_job (jobdef);
 \end{mycode}

 That example, as simple as it is, already shows most of what
 the attribute interface offers---that interface is really simple! 
 For some classes, the set of available attributes is fixed, and
 will never change.  That is also the case for the job
 description from the example above.  For other classes, such as for
 logical files (\T{saga::logical\_file}), the application programmer can
 specify arbitrary attributes -- their interpretations are, of
 course, up to the application again.

 Note that SAGA defines the known available attributes as static
 strings.  Thus, the above example is more safely written as:


 \begin{mycode}
  saga::job::description jobdef;

  std::vector <std::string> args;
  args.push_back ("2");
  
  namespace sja = saga::job::attributes;

  jobdef.set_attribute       (sja::description_executable,"/bin/sleep");
  jobdef.set_vector_attribute(sja::description_arguments, args);
 
  saga::job job = js.create_job (jobdef);
 \end{mycode}
 \subsubsection{Attribute Types}

  Although all attributes in SAGA are string-based, we distinguish
  between different types of attributes.  The available types and the
  prescribed formatting for the attribute values are summarized in
  Table~\ref{tab:attr-types}.  

   \UndefineShortVerb{\|}

\begin{table}
   \begin{tabular}{|l|l|l|}
    \hline
    \B{Type} & \B{Format} & \B{Example} \\
    \hline
    \T{String} & as in \T{printf ("\%s",   val);} & \T{Hello World} \\
    \T{Int}    & as in \T{printf ("\%lld", val);} & \T{123} \\
    \T{Float}  & as in \T{printf ("\%lld", val);} & \T{1.234E-4} \\
    \T{Time}   & as in \T{printf ("\%lld", val);} & \T{Mon Oct 20 11:31:54 1952} \\
    \T{Bool}   & \T{True} or \T{False}            & \T{True} \\
    \T{Enum}   & literal value of the enum        & \T{Done} \\
    \hline
  \end{tabular}
  \caption{Available types and the prescribed formatting for the attribute value}
  \label{tab:attr-types}
\end{table}

  \DefineShortVerb{\|}

  Trying to set an attribute which is, for its type, incorrectly
  formatted, will result in a 'BadParameter' exception.

  

