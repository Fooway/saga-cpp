
 \subsection{Quick Introduction}

  Tight security is arguably one of the most required features of Grid
  environments.  On the other hand, for the average end user, it is also one of the most
  annoying features.  That is mostly caused
  by the confusion about how security credentials are to be
  maintained, when and where they are valid, how to choose between
  them, etc.

  SAGA can alleviate only some of these problems---more standardization work outside of SAGA is required to be able to
  simplify credential management further.

  To understand SAGA's security model, one needs to consider two
  concepts: \I{sessions} and \I{contexts};  a |saga::context| simply
  represents one specific security credential; a |saga::session| can
  have multiple of these contexts attached, and is defined by the
  lifetime of the objects and operations using these credentials.

  By default, both |saga::session|s and |saga::context|s are
  invisible: a default session is always implicitly created on the
  SAGA call, and picks up all the default security credentials your
  SAGA implementation knows about, such as your default ssh keys, your
  default globus proxy (if it was initialized before), your default
  Unicore keyring, etc.  In most cases, that should allow writting
  applications which have no single line of code
  addressing security explicitly, which are still secure, as these
  credentials are used on adaptor level, as needed:

  \begin{mycode}[label=Secure gsiftp file copy]
  #include <saga/saga.hpp>

  int main ()
  {
    saga::url src ("ssh://remote.host.net/tmp/one.dat");
    saga::url tgt ("ssh://remote.host.net/tmp/two.dat");

    // do a file copy, using the default globus X509 proxy, in the
    // default session
    saga::filesystem::file f (src);
    f.copy (tgt);
  }
  \end{mycode}

  On the other hand, it allows simple means of control on which
  credentials are to be used for a given operation, as in the following
  example:
 
  \begin{mycode}[label=Secure gsiftp file copy]
  #include <saga/saga.hpp>

  int main ()
  {
    saga::url src ("ssh://remote.host.net/tmp/one.dat");
    saga::url tgt ("ssh://remote.host.net/tmp/two.dat");

    // create a new session, no default contexts are attached
    saga::session my_session;

    // create a new ssh context
    saga::context my_context ("ssh");

    // point ssh context to a specific ssh key
    my_context.set_attribute ("UserCert", 
                              "/home/username/.ssh/id_dsa.special");

    // add that context to my session.  The session will then 
    // contain *only* that explicitely added context.
    my_session.add_context (my_context);

    // create a new file object, in that session
    saga::filesystem::file f (my_session, src);

    // do a file copy, using the specified ssh key
    f.copy (tgt);
  }
  \end{mycode}

  \HINT{In most cases, the default session is what you want to use;
  if that does not work, try to convince your system administrator to
  configure SAGA so that the default session works!}

  Objects created from other objects inherit their session.
  Asynchronous operations are living in the session of their spawning
  objects.  A session does not die when going out of scope, but only
  when all associated objects and operations die/finish.


 \subsection{Reference}


 \subsection{Security Details}



