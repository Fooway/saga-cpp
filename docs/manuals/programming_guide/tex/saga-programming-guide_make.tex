
\subsection{Building your Application}

For Unix-like systems, a configure/make based build system is
provided.  That system can also be used to compile your SAGA
application, and is (briefly) described below.

Furthermore, your SAGA installation offers a tool named
|saga-config|, which can also be used to determine the relevant
compile parameters.

SAGA uses a gnu-make based build system.  It includes a number
of makefiles throughout the source tree, and the
|$SAGA_ROOT/make/| directory.  All these \I{make includes}\footnote{A make include is a makefile building block} get
installed into |$SAGA_LOCATION/share/saga/make/|\footnote{For
details on building and installing SAGA, please refer to the
SAGA installation manual.}.

When compiling your application, these make includes may provide
an easy starting point.  All you need to do is to set
|SAGA_LOCATION| to have it point to your SAGA installation
root, and to include |saga.application.mk|, which will add all
rules required to build a SAGA application.  The following
Makefile stub should get you started:


\begin{mycode}[label=Makefile for SAGA applications]
  SAGA_SRC  = $(wildcard *.cpp)
  SAGA_BIN  = $(SRC:%.cpp=%)

  include $(SAGA_LOCATION)/share/saga/make/saga.application.mk
\end{mycode}


This stub loads the make rules, etc., needed to build the application.
If the application needs additional include directories or libraries,
use the following syntax \I{after} the make includes:


 \begin{mycode}[label=Makefile: setting compiler/linker flags,
                firstnumber=last]
  SAGA_CPPFLAGS += -I/opt/super/include
  SAGA_LDFLAGS  += -L/opt/super/lib -lsuper
 \end{mycode}


Of course it is possible to build SAGA applications with custom
Makefiles.  The SAGA make includes can still, however, be used to
obtain the SAGA specific compiler:


 \begin{mycode}[label=Custom Makefile]
  SRC      = $(wildcard *.cpp)
  OBJ      = $(SRC:%.cpp=%.o)
  BIN      = $(SRC:%.cpp=%)
  
  CXX      = g++
  CXXFLAGS = -c -O3 -pthreads -I/opt/mpi/include
  
  LD       = $(CXX)
  LDFLAGS  = -L/usr/lib/ -lc
  
  include $(SAGA_LOCATION)/share/saga/make/saga.engine.mk
  include $(SAGA_LOCATION)/share/saga/make/saga.package.file.mk
  
  all: $(BIN)
  
  $(OBJ): %.o : %.cpp
  	$(CXX) $(CXXFLAGS) $(SAGA_CXXFLAGS) -o $@ $<
  
  $(BIN): % : %.o
  	$(LD)  $(LDFLAGS)  $(SAGA_LDFLAGS)  -o $@ $<
 \end{mycode}
 

|SAGA_CXXFLAGS| and |SAGA_LDFLAGS| contain only those options and settings
which are required to use SAGA.  You may want to use
|'make -n'| to print what the resulting make commands are, in order to
debug eventual incompatibilities between the SAGA compiler and linker
flags, and your own ones.

Yet another option is to use the output of saga-config for makefiles:

 \begin{mycode}[label=saga-config used in Makefile]
  SRC       = $(wildcard *.cpp)
  OBJ       = $(SRC:%.cpp=%.o)
  BIN       = $(SRC:%.cpp=%)
 
  CXX       = gcc
  CPPFLAGS  = -I/opt/mpi/include
  CXXFLAGS  = -c -O3 -pthreads
 
  LD        = $(CXX)
  LDFLAGS   = -L/opt/mpi/lib/ -lmpi
 
  CPPFLAGS += $(shell $(SAGA_LOCATION)/bin/saga-config --cppflags)
  CXXFLAGS += $(shell $(SAGA_LOCATION)/bin/saga-config --cxxflags)
  LDFLAGSS += $(shell $(SAGA_LOCATION)/bin/saga-config --ldflags)
 
  .default: $(BIN)
 
  $(BIN): % : %.cpp
 
  $(OBJ): %.o : %.cpp
    $(CXX) $(CPPFLAGS) $(CXXFLAGS) -o $@ $<
 
  $(BIN): % : %.o
    $(LD)  $(LDFLAGS) -o $@ $<
 \end{mycode}


\subsubsection*{Linkage Options}

 The SAGA libraries come in two flavours: \I{standard} and \I{lite}.
 The standard version is what you have observed in the examples
 above: your application is linked against the |libsaga_engine|, and
 against the available/required packages (|libsaga_package_abc|).  The
 adaptor libraries (|libsaga_adaptor_xyz|) are not linked to the
 application, but are loaded at runtime.  
 
 That standard version offers the most flexibility for your
 application, and allows it to adapt linkage to a wide range of use
 cases.  The runtime adaptor loading allows you to adapt your application
 to Grid middleware variations at runtime. That comes at a cost: at
 runtime, the shared library dependencies have to be resolved, and
 SAGA has to be correctly configured to find the adaptor libraries at
 runtime.  The former can be resolved by linking your application
 statically, but the SAGA runtime configuration cannot be avoided.

 That is the reason why we provide the second, 'lite' version of the
 SAGA library: it is a \I{single} shared library which contains the
 SAGA engine, all available packages, and a set of adaptors.  These
 adaptors are thus \I{not} loaded at runtime, but are linked at
 link-time.  Other adaptors can, however, still be loaded at runtime,
 as before.
 
 That way, no SAGA runtime configuration is required at all, as all
 dependencies are resolved on link-time.  The SAGA installation manual
 provides more information on the selection of adaptors to be included
 into the (|libsaga_lite|).  Note that this comes at a cost as well:
 your application binary will be larger, as all the adaptors are linked
 against it, whether they are needed or not.

 The SAGA make system supports linking against the (|libsaga_lite|)
 like this:

\begin{mycode}[label=Makefile for SAGA-Lite ]
  SAGA_SRC  = $(wildcard *.cpp)
  SAGA_BIN  = $(SRC:%.cpp=%)

  SAGA_USE_LITE = yes

  include $(SAGA_LOCATION)/share/saga/make/saga.application.mk
\end{mycode}


\subsection{Running your Application}

 Your SAGA application should have only few runtime
 dependencies.  First of all, it needs to find the shared
 libraries you used, and the shared libraries your SAGA
 installation is linked against.  That may include libraries
 required by the SAGA adaptors, e.g., globus libraries, openssl,
 etc.  Depending on your operating system, you may need to set
 the environment variables |LD_LIBRARY_PATH|,
 |DYLD_LIBRARYPATH|, or similar.

 After starting, SAGA reads a couple of initialization
 files to determine your specific set of adaptors, and several
 configuration options for these adaptors.  In general, you
 should not have the need to touch these ini files -- SAGA configuration
 and installation is supposed to be performed by the system
 administrator, not the end user.  In most cases, SAGA should be
 able to find these ini files automatically (the installation
 prefix is statically compiled into SAGA.  If that fails, SAGA
 searches the following locations (in this order) for the main
 saga.ini file:

    \shift |/etc/saga.ini|\\
    \shift |$SAGA_LOCATION/share/saga/saga.ini|\\
    \shift |$HOME/.saga.ini|\\
    \shift |$PWD/.saga.ini|\\
    \shift |$SAGA_INI|
 
 Those ini files point to the individual adaptor ini files,
 which then allow SAGA to load these adaptors.

 While running, the SAGA library can print log messages, of
 varying verbosity.  That output is controlled by the
 environment variable |SAGA_VERBOSE|.  The values are as
 follows:

   \shift |1: Critical|\\
   \shift |2: Error   |\\
   \shift |3: Warning |\\
   \shift |4: Info    |\\
   \shift |5: Debug   |\\
   \shift |6: Blurb   |
   
 When |SAGA_VERBOSE| is not set, the library is silent.

