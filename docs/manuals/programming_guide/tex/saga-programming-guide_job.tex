
 \subsection{Quick Introduction}


  For an overwhelming number of use cases, job submission and
  management is the most important aspect of Grid computing.
  The |saga::job| package provides the respective functionality.
  There are four classes:

  \begin{shortlist}

   \item |saga::job::description|\\
         describes the properties of a job to be submitted, 

   \item |saga::job::service|\\
         represents a service which accepts job descriptions for
         submission, 

   \item |saga::job::job|\\
         represents the submitted job, and

   \item |saga::job::self|\\
         which represents the application itself.

  \end{shortlist}


  The most basic code example is the following:

  %%%%%%%%%%%%%%%%%%%%%%%%%%%%%%%%%%%%%%%%%%%%%%%%
  \begin{mycode}[label=Job submission]
  #include <saga/saga.hpp>

  int main (int argc, char** argv)
  {
    saga::job::service js;
    saga::job::job j = js.run_job ("localhost", "/bin/sleep 3");
    j.wait ();
  
    return (0);
  }
  \end{mycode}

  Yes, it is that easy.  But what about the job's I/O, can we handle
  that as well?  And what about job state information \textless F2\textgreater?  Let's see:

  \begin{mycode}[label=Job submission and job I/O]
  saga::job::service js;
  saga::job::ostream in;
  saga::job::istream out;
  saga::job::istream err;
  
  saga::job::job j = js.run_job ("localhost",
                                 "/bin/date",
                                 in, out, err);

  saga::job::state state = j.get_state ();

  do 
  {
    char buffer[256];

    // get stdin
    out.read (buffer, sizeof (buffer));
    if ( out.gcount () > 0 )
    {
      std::cout << std::string (buffer, out.gcount ());
    }
    
    if ( out.fail () )
    {
      break;
    }

    usleep (10000);
    state = j.get_state ();

  } while ( state != saga::job::Done &&
            state != saga::job::Failed &&
            state != saga::job::Canceled )
  \end{mycode}

  That is basically the same as above, but we catch the job's I/O channel
  as it is getting created.  The remainder of the example code (lines
  12ff) is just watching the job state and the output stream.

  The |run_job()| method is just a shortcut.  It actually
  provides exactly the following functionality, just with less code,
  as the example below:

  \begin{mycode}[label=run\_job() expanded]
  namespace sa  = saga::attributes;
  namespace sja = saga::job::attributes;

  std::string exe ("/bin/date");

  std::vector <std::string> hosts;
  hosts.push_back ("localhost");

  saga::job::description jd;

  jd.set_attribute        (sja::description_interactive, sa::common_true);
  jd.set_attribute        (sja::description_executable, exe);
  jd.set_vector_attribute (sja::description_candidatehosts, hosts);

  saga::job::service js;

  saga::job::job j = js.create_job (jd);
  

  saga::job::ostream in  = j.get_stdin  ();
  saga::job::istream out = j.get_stdout ();
  saga::job::istream err = j.get_stderr ();
  
  // job is in 'New' state here, we need to run it
  j.run ();
  
  saga::job::state state = j.get_state ();

  do 
  {
    char buffer[256];

    // get stdin
    out.read (buffer, sizeof (buffer));
    if ( out.gcount () > 0 )
    {
      std::cout << std::string (buffer, out.gcount ());
    }
    if ( out.fail () )
    {
      break;
    }
    usleep (10000);
    state = j.get_state ();

  } while ( state != saga::job::Done &&
            state != saga::job::Failed &&
            state != saga::job::Canceled )
  \end{mycode}

  This long way exposes some additional details:

  \begin{shortlist}

   \item The |job_description| class allows specifying attributes of the
   job.\footnote{Only the \T{'Executable'} attribute is mandatory.}

   \item The |stdio| channels of a job can be obtained before the job
   is running, to avoid race conditions.

   \item Only |'Interactive'| jobs will provide |stdio| channels.

  \end{shortlist}


  Table~\ref{tab:jdattribs} shows the job description attributes
  available in SAGA.  Note that these attributes are closely related
  to the respective keys of the JSDL standard
  \cite{jsdl-spec,jsdl-spmd}.

  \UndefineShortVerb{\|}
  
  \begin{table}[h!]
   \centering
   \begin{tabular}{|l|l|l|}
    \hline
      Executable          & string          & what executable to run \\
      Arguments           & list of strings & the job arguments      \\
      CandidateHosts      & list of strings & where to run           \\
      SPMDVariation       & string          & MPI type (if any)      \\
      TotalCPUCount       & int             & \# CPUs, total         \\
      NumberOfProcesses   & int             & \# processes, total    \\
      ProcessesPerHost    & int             & \# processes per host  \\
      ThreadsPerProcess   & int             & \# threads per process \\
      Environment         & list of strings & environment vars       \\
      WorkingDirectory    & string          & working directory      \\
      Interactive         & bool            & usage type             \\
      Input               & string          & stdin source file      \\
      Output              & string          & stdout log file        \\
      Error               & string          & stderr log file        \\
      FileTransfer        & list of strings & file staging           \\
      Cleanup             & bool            & cleanup-after-run      \\
      JobStartTime        & datetime        & when to start job      \\
      TotalCPUTime        & int             & how long will job run  \\
      TotalPhysicalMemory & int             & memory the job needs   \\
      CPUArchitecture     & string          & architecture to run on \\
      OperatingSystemType & string          & OS to run on           \\
      Queue               & string          & queue to submit to     \\
      JobContact          & list of strings & notification contact   \\
    \hline
   \end{tabular}
   \caption{Job description attributes available in SAGA.}
   \label{tab:jdattribs}
   \end{table}

  \DefineShortVerb{\|}

  The available |SPMDVariation| values are: |MPI|, |GridMPI|,
  |IntelMPI| |LAM-MPI|, |MPICH1|, |MPICH2|, |MPICH-GM|, |MPICH-MX|
  |MVAPICH|, |MVAPICH2|, |OpenMP|, |POE|, |PVM| and |None|.  Other
  arbitrary values are allowed, and interpreted by the backend.  This
  attribute indirectly dtermines which |mpirun| executable should be
  called for starting the application, and which parameters it will
  accept.
  

  % FIXME: should enumerate examples!
  Line 10 in the job I/O example above obtains the job state:

  \shift |saga::job::state state = j.get_state ();|

  That implies that |saga::job| is a stateful object.  The job state
  model is shown in Figure~\ref{fig:job_states}, which also shows what
  method calls cause which state transitions.

  \myfig{job_states}{\footnotesize The SAGA job state model.}


 \subsection{Reference}

   \begin{mycode}[label=Prototypes: saga::job]
  namespace saga
  {
    namespace job 
    {
      typedef char const * char ccc_type;
  
      namespace attributes 
      {
        // job description atributes
        ccc_type description_executable          = "Executable";
        ccc_type description_arguments           = "Arguments";
        ccc_type description_environment         = "Environment";
        ccc_type description_workingdirectory    = "WorkingDirectory";
        ccc_type description_interactive         = "Interactive";
        ccc_type description_input               = "Input";
        ccc_type description_output              = "Output";
        ccc_type description_error               = "Error";
        ccc_type description_filetransfer        = "FileTransfer";
        ccc_type description_cleanup             = "Cleanup";
        ccc_type description_jobstarttime        = "JobStartTime";
        ccc_type description_totalcputime        = "TotalCPUTime";
        ccc_type description_totalphysicalmemory = "TotalPhysicalMemory";
        ccc_type description_cpuarchitecture     = "CPUArchitecture";
  
        ccc_type cpuarchitecture_sparc           = "sparc";
        ccc_type cpuarchitecture_powerpc         = "powerpc";
        ccc_type cpuarchitecture_x86             = "x86";
        ccc_type cpuarchitecture_x86_32          = "x86_32";
        ccc_type cpuarchitecture_x86_64          = "x86_64";
        ccc_type cpuarchitecture_parisc          = "parisc";
        ccc_type cpuarchitecture_mips            = "mips";
        ccc_type cpuarchitecture_ia64            = "ia64";
        ccc_type cpuarchitecture_arm             = "arm";
        ccc_type cpuarchitecture_other           = "other";
  
        ccc_type description_operatingsystemtype = "OperatingSystemType";
        ccc_type description_candidatehosts      = "CandidateHosts";
        ccc_type description_queue               = "Queue";
        ccc_type description_jobcontact          = "JobContact";
        ccc_type description_spmdvariation       = "SPMDVariation";
        ccc_type description_totalcpucount       = "TotalCPUCount";
        ccc_type description_numberofprocesses   = "NumberOfProcesses";
        ccc_type description_processesperhost    = "ProcessesPerHost";
        ccc_type description_threadsperprocess   = "ThreadsPerProcess";
      }
  
      class description 
          : public saga::object,
            public saga::attributes
      {
        public:
          description (void);
         ~description (void);
      }; 
  
  
      class service 
          : public saga::object
      {
        public:
          service  (saga::session const & s, 
                    saga::url             rm = "");
          service  (saga::url             rm = "");
         ~service  (void);
  
          saga::job::job create_job (description job_desc);
  
          saga::job::job run_job    (std::string   hostname,
                                     std::string   commandline,
                                     ostream     & stdin_stream, 
                                     istream     & stdout_stream, 
                                     istream     & stderr_stream);
          std::vector <std::string> 
                          list      (void); 
          saga::job::job  get_job   (std::string   job_id); 
          saga::job::self get_self  (void);
      };
  
  
      namespace attributes 
      {
        //  read only job attributes
        ccc_type jobid            = "JobID";
        ccc_type executionhosts   = "ExecutionHosts";
        ccc_type created          = "Created";
        ccc_type started          = "Started";
        ccc_type finished         = "Finished";
        ccc_type workingdirectory = "WorkingDirectory";
        ccc_type exitcode         = "ExitCode";
        ccc_type termsig          = "Termsig";
      }
  
      namespace metrics 
      { 
        // job metrics
        ccc_type statedetail      = "job.StateDetail";
        ccc_type signal           = "job.Signal";
        ccc_type cputimelimit     = "job.CPUTimeLimit";
        ccc_type memoryuse        = "job.MemoryUse";
        ccc_type vmemoryuse       = "job.VmemoryUse";
        ccc_type performance      = "job.Performance";
      }
  
      enum state 
      {
        // job state
        Unknown   =  saga::task_base::Unknown,   // -1
        New       =  saga::task_base::New,       //  0
        Running   =  saga::task_base::Running,   //  1
        Failed    =  saga::task_base::Failed,    //  2
        Done      =  saga::task_base::Done,      //  3
        Canceled  =  saga::task_base::Canceled,  //  4
        Suspended =  5
      };    
  
      class job : public saga::task,
                  public saga::attributes,
                  public saga::permissions,
                  public saga::sync
      {
        public:
          // no constructor
          ~job (void);

          job & operator= (saga::object const & o);

          std::string get_job_id      (void);
          state       get_state       (void);
          description get_description (void);
          ostream     get_stdin       (void);
          istream     get_stdout      (void);
          istream     get_stderr      (void);

          void        suspend         (void);
          void        resume          (void);

          void        checkpoint      (void);
          void        migrate         (description job_desc);
          void        signal          (int         signal);
      };
    }
  }
   \end{mycode}


 \subsection{Job Details}

  The SAGA Job API covers four classes: |saga::job_description|,
  \T{saga::job\_-} \T{service}, |saga::job|, and |saga::job_self|.
  The job description class is nothing more than a container for a
  well-defined set of attributes, which, using
  JSDL~\cite{jsdl-spec,jsdl-spmd} based keys, defines the job to be
  started, and its runtime and resource requirements.  The job server
  represents a resource management endpoint which allows the starting
  and insertion of jobs. 
  
  The job class itself is central to the API, and represents an
  application instance running under the management of a
  resource manager.  The |job_self| class |IS-A| job.  Its
  purpose is to represent the current SAGA application, which
  allows for a number of use cases with applications that
  actively interact with the grid infrastructure, for example, to
  migrate itself, or to set new job attributes.
  
  The job class inherits the |saga::task| class (Sec.~\ref{sec:tasks}),
  and uses its methods to |run()|, to |wait()| for, and to
  |cancel()| jobs.  The inheritance also allows for the
  management of large numbers of jobs in task containers.
  Additional methods provided by the |saga::job| class relate to
  the |Suspended| state (which is not available on tasks), and
  provide access to the job's standard I/O streams, and to more
  detailed status information.  The |saga::job| iostreams are
  available as |saga::job::istream| and |saga::job::ostream|
  instances, which all inherit from |std::stream|.


  \subsubsection{Job State Model}
  
    The SAGA job state diagram is shown in
    Figure~\ref{fig:job_states}.  It is an extension of the
    |saga::task| state diagram (Figure~\ref{fig:task_states}),
    and extends the state diagram with a |'Suspended'| state, which the
    job can enter/leave using the |suspend()/resume()| calls.  

    \myfig{job_states}{\footnotesize The SAGA job state model
    extends the SAGA task state model with 
    a 'Suspended' state, and additional transitions.}

    SAGA maps the native back-end state
    model onto the SAGA state model.  The SAGA state model
    is simple enough to allow a straightforward mapping
    in most cases.  For some applications, access to the native
    back-end state model is useful, though---for that reason, an
    additional metric named |'StateDetail'| allows to query the
    native job state.  

    State details in SAGA are formatted as follows:

    \shift |'<model>:<state>'|

    with valid models being "BES", "DRMAA", or other
    implementation specific models.  For example, a state detail
    for the BES state 'StagingIn' would be rendered as
    |'BES:StagingIn'|, and would be a substate of
    \T{Running}.  If no state details are available, the metric
    is still available, but it has always an empty string value.


    
  \subsubsection{File Transfer Specifications}
  
  The syntax of a file transfer directive for the job
  description is modeled on the LSF syntax (LSF stands for
  \I{Load Sharing Facility}, a commercial job scheduler by
  Platform Computing). It has the general syntax:

    \shift |local_file operator remote_file|

    Both the |local_file| and the |remote_file| can be URLs. If
    they are not URLs, but full or relative pathnames, then the
    |local_file| is relative to the host where the submission is
    executed, and the |remote_file| is evaluated on the
    execution host of the job.

    The operator is one of the following four:

    \begin{tabular}{lp{0.75\textwidth}}

      \shift |'>'| &  copies the local file to the remote file
      before the job starts. Overwrites the remote file if it
      exists.\\

      \shift |'>>'| & copies the local file to the remote file
      before the job starts. Appends to the remote file if it
      exists.\\

      \shift |'<'| & copies the remote file to the local file
      after the job finishes. Overwrites the local file if it
      exists.\\

      \shift |'<<'| & copies the remote file to the local file
      after the job finishes. Appends to the local file if it
      exists.\\

    \end{tabular}

  
  \subsubsection{Command Line Specification}

    The |run_job()| method of the |saga::job_service| class
    accepts a string parameter which constitutes a command line
    to be executed on a remote resource.  The parsing of that
    command line follows the following rules:

    \begin{itemize}
      \item Elements are either delimited by white space (a space, or a tab), 
            or are delimited by not-escaped double quotes.
      \item The escape character for double quotes is the
            backslash (which is also used to escape the backslash
            itself).
      \item The first element is used as the executable name;
            all other elements are treated as job arguments.
    \end{itemize}

    The following command line parameter

    \shift |/bin/cp "/tmp/my file with spaces" /data/|

    is thus invoking the executable |/bin/cp|, with two
    arguments.


  
  \subsubsection{Job Identifiers}
  
    The SAGA \T{JobID} is treated as an opaque string in the
    SAGA API, and is structured as: 

      \shift |'[backend url]-[native id]'|

    For example, a job submitted to the host |remote.host.net|
    via |ssh| (whose daemon runs on port |22|), and having the
    POSIX PID |1234|, would get the job id:

      \shift |'[ssh://remote.host.net:22/]-[1234]'|

    The implementation may free the resources used for the job,
    and hence may invalidate a \T{JobID} after a successful
    wait on the job, and after the application received the job
    status information at least once.

    A JobID may be unknown until the job enters the \T{Running}
    state, as the back-end will often not assign IDs to jobs
    which are not yet running.  In such cases, the value of the
    \T{JobID} attribute is empty.  The job will, however,
    retain its JobID after it enters a final state.
      

